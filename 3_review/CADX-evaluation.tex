\subsection{Evaluation measures} \label{subsec:chp3:img-clas:eval-mea}

\begin{table}
  \caption{Overview of the evaluation metrics used in \acs*{cad} systems.}
  \begin{tabularx}{\textwidth}{@{}l >{\raggedleft\arraybackslash}X@{}}
    \toprule
    \textbf{Evaluation metrics} & \textbf{References} \\
    \midrule
    \quad Accuracy & \cite{Artan2009,Artan2010,Liu2009,Sung2011,Tiwari2012} \\
    \quad Sensitivity - Specificity & \cite{Artan2009,Artan2010,Giannini2013,Liu2009,Lopes2011,Mazzetti2011,Ozer2009,Ozer2010,Parfait2012,Peng2013,Tiwari2008,Tiwari2009,Viswanath2008,Viswanath2008a,trigui2016classification,trigui2017automatic,samarasinghe2016semi,cameron2014multiparametric,cameron2016maps,khalvati2015automated} \\
    \quad \acs*{roc} - \acs*{auc} & \cite{Ampeliotis2008,Antic2013,Chan2003,Giannini2013,Kelm2007,Langer2009,Liu2013,Lopes2011,Lv2009,Matulewicz2013,Mazzetti2011,Niaf2011,Niaf2012,Peng2013,Tiwari2009a,Tiwari2010,Tiwari2012,Tiwari2013,Viswanath2009,Viswanath2011,Viswanath2012,Vos2008,Vos2008a,Vos2010,giannini2015fully,lehaire2014computer,rampun2015classifying,rampun2015computer,rampun2016computer,rampun2016computerb,rampun2016quantitative} \\
    \quad \acs*{froc} & \cite{Litjens2011,Litjens2012,Vos2012} \\
    \quad Dice's coefficient & \cite{Artan2009,Artan2010,Liu2009,Ozer2009} \\
    \bottomrule
  \end{tabularx}
\label{tab:evatec}
\end{table}

Several metrics are used in order to assess the performance of a classifier and
are summarized in \acs{tab}~\ref{tab:evatec}.
Voxels in the \ac{mri} image are classified into healthy or malign tissue and
compared with a ground-truth.
This allows to compute a confusion matrix by counting true positive (TP), true
negative (TN), false positive (FP), and false negative (FN) samples.
From this analysis, different statistics are extracted.

The first statistic used is the accuracy which is computed as the ratio of true
detection to the number of samples.
However, depending on the strategy employed in the \ac{cad} work-flow, this
statistic is highly biased by a high number of true negative samples which
boost the accuracy score overestimating the actual performance of the
classifier.
That is why, the most common statistics computed are sensitivity and
specificity defined in \acs{eq}\,\eqref{eq:sens} and \acs{eq}\,\eqref{eq:spec},
respectively.
The metrics give a full overview of the performance of the classifier.

\begin{equation}
  \text{SE} = \frac{\text{TP}}{\text{TP} + \text{FN}} \ ,
  \label{eq:sens}
\end{equation}

\begin{equation}
  \text{SP} = \frac{\text{TN}}{\text{TN} + \text{FP}} \ .
  \label{eq:spec}
\end{equation}

These statistics are also used to compute the \ac{roc} curves~\cite{Metz2006},
which give information about voxel-wise classification.
This analysis represents graphically the sensitivity as a function of $(1 -
\text{specificity})$, which is in fact the false positive rate, by varying the
discriminative threshold of the classifier.
By varying this threshold, more true negative samples are found but often at
the cost of detecting more false negatives.
However, this fact is interesting in \ac{cad} since it is possible to obtain a
high sensitivity and to ensure that no cancers are missed even if more false
alarms have to be investigated or the opposite.
A statistic derived from \ac{roc} analysis is the \acf{auc} which corresponds
to the area under the \ac{roc} and is a measure used to make comparisons
between models.

The \acf{froc} extends the \ac{roc} analysis but to a lesion-based level.
The same confusion matrix is computed where the sample are not pixels but
lesions.
However, it is important to define what is a true positive sample in that case.
Usually, a lesion is considered as a true positive sample if the region
detected by the classifier overlaps ``sufficiently'' the one delineated in the
ground-truth.
However, ``sufficiently'' is a subjective measure defined by each researcher
and can correspond to one pixel only.
However, an overlap of \SIrange{30}{50}{\percent} is usually adopted.
Finally, in addition to the overlap measure, the Dice's coefficient is often
computed to evaluate the accuracy of the lesion localization.
This coefficient consists of the ratio between twice the number of pixels in
common and the sum of the pixels of the lesions in the ground-truth $\text{GT}$
and the output of the classifier $\text{S}$, defined as shown in
\acs{eq}\,\eqref{eq:dice}.

\begin{equation}
  Q_D = \frac{2 | \text{GT} \cap \text{S} |}{| \text{GT} | + | \text{S} |} \ .
  \label{eq:dice}
\end{equation}
