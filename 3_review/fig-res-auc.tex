\begin{figure}
  \centering
 % \hspace{\fill}
  \subfloat[]{
    \label{fig:auc15}
    \begin{tikzpicture}[scale=.7,every node/.style={scale=0.7}]

      \def\labels{
        {\color{blue}\cite{Ampeliotis2008}},
        {\color{blue}\cite{Antic2013}},
        {\color{blue}\cite{Chan2003}},
        {\color{blue}\cite{Giannini2013}},
        {\color{blue}\cite{Langer2009}},
        {\color{blue}\cite{Lopes2011}},
        {\color{blue}\cite{Lv2009}},
        {\color{blue}\cite{Niaf2011}},
        {\color{blue}\cite{Niaf2012}},
        {\color{blue}\cite{Tiwari2009a}},
        {\color{blue}\cite{Tiwari2010}},
        {\color{blue}\cite{Tiwari2012}},
        {\color{blue}\cite{Tiwari2013}},
        {\color{blue}\cite{Vos2008}},
        {\color{blue}\cite{Vos2010}},
        {\color{blue}\cite{lehaire2014computer}},
        {\color{blue}\cite{giannini2015fully}},
        {\color{blue}\cite{Mazzetti2011}},
        {\color{blue}\cite{Puech2009}},
        {\color{blue}\cite{Vos2008a}},
        {\color{blue}\cite{rampun2016computerb}},
        {\color{blue}\cite{rampun2016computer}}
      }

      \def\reward{90,94,84,87,71,93,97,87,87,84,91,90,85,91,97,75,91,92,77,90,93,93}
      \def\dbSize{25,53,15,10,25,27,55,23,30,15,19,36,29,29,29,35,56,29,100,10,45,45}
      \def\dbClass{1,2,1,1,1,1,2,1,1,1,1,1,1,1,1,1,2,1,3,1,2,2}
      \def\cZoom{3}
      \def\percentageLabelAngle{90}
      \def\nbeams{22}
      \pgfmathsetmacro\beamAngle{(360/\nbeams)}
      \pgfmathsetmacro\halfAngle{(180/\nbeams)}
      \pgfmathsetmacro\globalRotation{\halfAngle}

      % draw the radiants
      \foreach \n  [count=\ni] in \labels
      {
        \pgfmathsetmacro\cAngle{{(\ni*(360/\nbeams))+\globalRotation}}
        \draw(\cAngle:{\cZoom*1.15})  node[fill=white] {\n};
        \draw [thin] (0,0) -- (\cAngle:{\cZoom*1}) ;

      }

      % draw the % rings
      \foreach \x in {12.5,25, ...,100}
      \draw [thin,color=gray!50] (0,0) circle [radius={\cZoom*\x/100}];

      \foreach \x in {50,75,100}
      {
        \draw [thin,color=black!50] (0,0) circle [radius={\cZoom/100*\x}];
        \foreach \a in {0, 180} \draw ({\percentageLabelAngle+\a}:{\cZoom*0.01*\x}) node  [inner sep=0pt,outer sep=0pt,fill=white,font=\fontsize{8}{8.5}\selectfont]{$\x\%$};
      }

      % draw the path of the percentages
      \def\aux{{\reward}}
      \pgfmathsetmacro\origin{\aux[\nbeams-1]}
      \draw [semiAuto, thick] (\globalRotation:{\cZoom*\origin/100}) \foreach \n  [count=\ni] in \reward { -- ({(\ni*(360/\nbeams))+\globalRotation}:{\cZoom*\n/100}) } ;

      % label all the percentags
      \foreach \n [count=\ni] in \dbSize
      {
        \pgfmathsetmacro\cAngle{{(\ni*(360/\nbeams))+\globalRotation}}
        \pgfmathsetmacro\nreward{\aux[\ni-1]}
        \draw (\cAngle:{\cZoom*1.5}) node[align=center] {{\color{semiAuto}\nreward $\%$} \\ {\color{red}\n} };
      } ;

      % draw the database rose
      \def\dbScale{\9}
      \foreach \n [count=\ni] in \dbClass
      \filldraw[fill=red!20!white, draw=red!50!black]
      (0,0) -- ({\ni*(360/\nbeams)-\halfAngle+\globalRotation}:{\cZoom*\n/9}) arc ({\ni*(360/\nbeams)-\halfAngle+\globalRotation}:{\ni*(360/\nbeams)+\halfAngle+\globalRotation}:{\cZoom*\n/9}) -- cycle;
      \foreach \x in {1,2,3}
      \draw [thin,color=red!50!black,dashed] (0,0) circle [radius={\cZoom*\x/9}];

      %% draw the domain of each class
      \def\puta{17/0/{Multiparametric},
        5/17/{Monoparametric}}

      \foreach \numElm/\contadorQueNoSeCalcular/\name [count=\ni] in \puta
      {

        \pgfmathsetmacro\initialAngle{(\contadorQueNoSeCalcular*\beamAngle)+\halfAngle+\globalRotation}
        \pgfmathsetmacro\finalAngle  {((\numElm+\contadorQueNoSeCalcular)*\beamAngle)+\halfAngle+\globalRotation}
        \pgfmathsetmacro\l  {\cZoom*1.65+.3pt}
        \draw (\initialAngle:{\cZoom*1.7}) -- (\initialAngle:{\cZoom*1.1});
        \draw [ |<->|,>=latex] (\initialAngle:\l) arc (\initialAngle:\finalAngle:\l) ;
        \pgfmathsetmacro\r  {\cZoom*1.65+.45pt}
        {\draw [decoration={raise=4pt,text along path,text={\name},text align={center}},decorate] (\finalAngle:\r) arc (\finalAngle:\initialAngle:\r);}
      }

    \end{tikzpicture}}\\
  \subfloat[]{
    \label{fig:auc30}
    \begin{tikzpicture}[scale=.7,every node/.style={scale=0.7}]

      \def\labels{
        {\color{blue}\cite{Litjens2014}},
        {\color{blue}\cite{Liu2013}},
        {\color{blue}\cite{Peng2013}},
        {\color{blue}\cite{Viswanath2009}},
        {\color{blue}\cite{Viswanath2011}},
        {\color{blue}\cite{khalvati2015automated}},
        {\color{blue}Our},
        {\color{blue}\cite{Viswanath2012}}
      }

      \def\reward{94,83,95,82,77,90,84,80}
      \def\dbSize{347,54,48,6,12,20,19,22}
      \def\dbClass{3,2,2,1,1,1,1,1}
      \def\cZoom{3}
      \def\percentageLabelAngle{90}
      \def\nbeams{8}
      \pgfmathsetmacro\beamAngle{(360/\nbeams)}
      \pgfmathsetmacro\halfAngle{(180/\nbeams)}
      \pgfmathsetmacro\globalRotation{\halfAngle}

      % draw the radiants
      \foreach \n  [count=\ni] in \labels
      {
        \pgfmathsetmacro\cAngle{{(\ni*(360/\nbeams))+\globalRotation}}
        \draw(\cAngle:{\cZoom*1.15})  node[fill=white] {\n};
        \draw [thin] (0,0) -- (\cAngle:{\cZoom*1}) ;

      }

      % draw the % rings
      \foreach \x in {12.5,25, ...,100}
      \draw [thin,color=gray!50] (0,0) circle [radius={\cZoom*\x/100}];

      \foreach \x in {50,75,100}
      {
        \draw [thin,color=black!50] (0,0) circle [radius={\cZoom/100*\x}];
        \foreach \a in {0, 180} \draw ({\percentageLabelAngle+\a}:{\cZoom*0.01*\x}) node  [inner sep=0pt,outer sep=0pt,fill=white,font=\fontsize{8}{8.5}\selectfont]{$\x\%$};
      }

      % draw the path of the percentages
      \def\aux{{\reward}}
      \pgfmathsetmacro\origin{\aux[\nbeams-1]}
      \draw [semiAuto, thick] (\globalRotation:{\cZoom*\origin/100}) \foreach \n  [count=\ni] in \reward { -- ({(\ni*(360/\nbeams))+\globalRotation}:{\cZoom*\n/100}) } ;

      % label all the percentags
      \foreach \n [count=\ni] in \dbSize
      {
        \pgfmathsetmacro\cAngle{{(\ni*(360/\nbeams))+\globalRotation}}
        \pgfmathsetmacro\nreward{\aux[\ni-1]}
        \draw (\cAngle:{\cZoom*1.5}) node[align=center] {{\color{semiAuto}\nreward $\%$} \\ {\color{red}\n} };
      } ;

      % draw the database rose
      \def\dbScale{\9}
      \foreach \n [count=\ni] in \dbClass
      \filldraw[fill=red!20!white, draw=red!50!black]
      (0,0) -- ({\ni*(360/\nbeams)-\halfAngle+\globalRotation}:{\cZoom*\n/9}) arc ({\ni*(360/\nbeams)-\halfAngle+\globalRotation}:{\ni*(360/\nbeams)+\halfAngle+\globalRotation}:{\cZoom*\n/9}) -- cycle;
      \foreach \x in {1,2,3}
      \draw [thin,color=red!50!black,dashed] (0,0) circle [radius={\cZoom*\x/9}];

      %% draw the domain of each class
      \def\puta{7/0/{Multiparametric},
        1/7/{Monoparametric}}

      \foreach \numElm/\contadorQueNoSeCalcular/\name [count=\ni] in \puta
      {

        \pgfmathsetmacro\initialAngle{(\contadorQueNoSeCalcular*\beamAngle)+\halfAngle+\globalRotation}
        \pgfmathsetmacro\finalAngle  {((\numElm+\contadorQueNoSeCalcular)*\beamAngle)+\halfAngle+\globalRotation}
        \pgfmathsetmacro\l  {\cZoom*1.65+.3pt}
        \draw (\initialAngle:{\cZoom*1.7}) -- (\initialAngle:{\cZoom*1.1});
        \draw [ |<->|,>=latex] (\initialAngle:\l) arc (\initialAngle:\finalAngle:\l) ;
        \pgfmathsetmacro\r  {\cZoom*1.65+.45pt}
        {\draw [decoration={raise=4pt,text along path,  text={\name},text align={center}},decorate] (\finalAngle:\r) arc (\finalAngle:\initialAngle:\r);}
      }

    \end{tikzpicture}
 }
  %\hspace{\fill}
  \caption[Results comparison from the state-of-the-art in terms of \acs*{auc}.]{Numerical and graphical comparison of the results in terms of \acs*{auc} for \SI{1.5}{\tesla}~\subref{fig:auc15} and \SI{3}{\tesla}~\subref{fig:auc30} \acs*{mri} scanners. The {\color{semiAuto}green} value represents the metric and are graphically reported in the {\color{semiAuto}green} curve in the center of the figure. The {\color{red}red} value and areas correspond to the number of patients in the dataset. The numbers between brackets in {blue\color{blue}} correspond to the reference as reported in \acs{tab}~\ref{tab:sumpap}.}
  \label{fig:auc}
\end{figure}
